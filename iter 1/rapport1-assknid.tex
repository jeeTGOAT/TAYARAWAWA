\documentclass[12pt,a4paper]{article}
\usepackage[utf8]{inputenc}
\usepackage[T1]{fontenc}
\usepackage[french]{babel}
\usepackage{graphicx}
\usepackage{amsmath}
\usepackage{amssymb}
\usepackage{listings}
\usepackage{color}
\usepackage{hyperref}
\usepackage{enumitem}
\usepackage{geometry}
\usepackage{tikz}
\usepackage{pgf-umlsd}
\usepackage{fancyhdr}

\geometry{a4paper, margin=2.5cm}

% Fix the header height
\setlength{\headheight}{16.35004pt}
\addtolength{\topmargin}{-0.996pt}

\pagestyle{fancy}
\fancyhf{}
\fancyhead[L]{\includegraphics[height=12pt]{n7.png}}
\fancyhead[C]{Équipe KL-4}
\fancyhead[R]{\thepage}
\fancyfoot[C]{Simulateur d'Avion - 2025}

\title{
    \includegraphics[width=0.4\textwidth]{n7.png}\\[1cm]
    \textbf{Rapport d'Itération 1 - Projet Simulateur d'Avion}
}
\author{ASSKNID Walid - Équipe KL-4}
\date{14 Avril 2025}

\begin{document}

\maketitle
\thispagestyle{empty}

\section{Introduction}
Ce rapport présente les activités réalisées lors de la première itération du projet de simulateur d'avion par l'équipe KL-4. Cette itération a permis de mettre en place les bases du projet et d'implémenter les premières fonctionnalités essentielles.

\section{Activités réalisées}

\subsection{Analyse des besoins}
\begin{itemize}
    \item \textbf{Analyse du document de fonctionnalités} (100\%)
    \begin{itemize}
        \item Lecture et compréhension des fonctionnalités attendues
        \item Identification des fonctionnalités prioritaires pour la première itération
    \end{itemize}
    
    \item \textbf{Conception initiale} (100\%)
    \begin{itemize}
        \item Création du diagramme de classes UML
        \item Définition de l'architecture générale de l'application
    \end{itemize}
\end{itemize}

\subsection{Développement}
\begin{itemize}
    \item \textbf{Interface utilisateur} (90\%)
    \begin{itemize}
        \item Création de la page d'accueil avec menu principal
        \item Implémentation des boutons de navigation
        \item Mise en place du système de transition entre les écrans
    \end{itemize}
    
    \item \textbf{Simulation de base} (80\%)
    \begin{itemize}
        \item Implémentation de la classe Aircraft pour représenter les avions
        \item Création du panneau de simulation avec affichage de l'avion
        \item Mise en place des contrôles de base (accélération, décélération, rotation)
        \end{itemize}
    
    \item \textbf{Gestion des événements} (70\%)
    \begin{itemize}
        \item Implémentation des écouteurs de clavier pour contrôler l'avion
        \item Mise en place des écouteurs de souris pour la sélection d'avions
    \end{itemize}
\end{itemize}

\subsection{Tests}
\begin{itemize}
    \item \textbf{Tests de base} (60\%)
    \begin{itemize}
        \item Vérification du bon fonctionnement du menu principal
        \item Tests des contrôles de l'avion
    \end{itemize}
\end{itemize}

\section{Problèmes rencontrés}
\begin{itemize}
    \item Difficulté à gérer correctement les transitions entre les écrans
    \item Problèmes de performance lors du rendu des avions
    \item Gestion des événements de clavier qui nécessite une attention particulière
\end{itemize}

\section{Prochaines étapes}
Pour la prochaine itération, les objectifs sont :
\begin{itemize}
    \item Finaliser l'interface utilisateur
    \item Améliorer la simulation avec l'ajout de nuages et d'effets météorologiques
    \item Implémenter la gestion des collisions
    \item Ajouter des fonctionnalités de vol planifié
\end{itemize}

\section{Conclusion}
Cette première itération a permis de mettre en place les bases du projet et d'implémenter les fonctionnalités essentielles. Les progrès réalisés sont encourageants, mais il reste encore du travail à faire pour atteindre les objectifs fixés dans le document de fonctionnalités.

\end{document} 