\documentclass[12pt,a4paper]{article}
\usepackage[utf8]{inputenc}
\usepackage[T1]{fontenc}
\usepackage[french]{babel}
\usepackage{graphicx}
\usepackage{xcolor}
\usepackage{hyperref}
\usepackage{enumitem}
\usepackage{geometry}
\usepackage{fancyhdr}
\usepackage{tcolorbox}
\usepackage{fontawesome}

\geometry{a4paper, margin=2.5cm}

% Fix the header height
\setlength{\headheight}{16.35004pt}
\addtolength{\topmargin}{-0.996pt}

\pagestyle{fancy}
\fancyhf{}
\fancyhead[L]{\includegraphics[height=12pt]{n7.png}}
\fancyhead[C]{Manuel Utilisateur}
\fancyhead[R]{\thepage}
\fancyfoot[C]{Simulateur d'Avion - 2025}

% Configuration des boîtes colorées
\tcbset{
    note/.style={
        colback=blue!5,
        colframe=blue!75!black,
        fonttitle=\bfseries,
        title=Note:
    },
    warning/.style={
        colback=red!5,
        colframe=red!75!black,
        fonttitle=\bfseries,
        title=Attention:
    },
    tip/.style={
        colback=green!5,
        colframe=green!75!black,
        fonttitle=\bfseries,
        title=Astuce:
    }
}

\title{
    \includegraphics[width=0.4\textwidth]{n7.png}\\[1cm]
    \Huge\textbf{Manuel Utilisateur}\\[0.5cm]
    \Large\textbf{Simulateur d'Avion}\\[0.5cm]
    \large Version 1.0
}
\author{ASSKNID Walid - Équipe KL-4}
\date{\today}

\begin{document}

\maketitle
\thispagestyle{empty}

\tableofcontents
\newpage

\section{Introduction}
Ce manuel vous guidera dans l'utilisation du simulateur d'avion. Il vous permettra de comprendre les fonctionnalités de base et avancées du logiciel.

\section{Configuration requise}
\begin{itemize}
    \item Java Runtime Environment (JRE) 11 ou supérieur
    \item Système d'exploitation : Windows, Linux ou MacOS
    \item Mémoire RAM : 4 Go minimum
    \item Résolution d'écran minimale : 1280x720
    \item Souris et clavier
\end{itemize}

\section{Installation}
\begin{enumerate}
    \item Téléchargez le fichier \texttt{simulateur-avion.jar}
    \item Double-cliquez sur le fichier pour lancer l'application
    \item Alternativement, vous pouvez lancer l'application en ligne de commande :
    \begin{verbatim}
    java -jar simulateur-avion.jar
    \end{verbatim}
\end{enumerate}

\begin{tcolorbox}[note]
Assurez-vous que Java est correctement installé sur votre système avant de lancer l'application.
\end{tcolorbox}

\section{Interface principale}

\subsection{Menu principal}
Le menu principal propose les options suivantes :
\begin{itemize}
    \item \textbf{Nouvelle simulation} : Démarre une nouvelle session
    \item \textbf{Quitter} : Ferme l'application
\end{itemize}

\subsection{Interface de simulation}
L'interface de simulation comprend :
\begin{itemize}
    \item Zone de visualisation principale
    \item Panneau de contrôle (gauche)
    \item Informations de vol (droite)
    \item Barre d'état (bas)
\end{itemize}

\section{Contrôles de base}

\subsection{Contrôle de la vue}
\begin{itemize}
    \item \textbf{Souris} : 
    \begin{itemize}
        \item Clic gauche : Sélectionner un avion
        \item Clic droit : Menu contextuel
        \item Molette : Zoom avant/arrière
    \end{itemize}
    \item \textbf{Clavier} :
    \begin{itemize}
        \item Flèches : Déplacer la vue
        \item + / - : Zoom avant/arrière
        \item Espace : Centrer la vue
    \end{itemize}
\end{itemize}

\subsection{Contrôle des avions}
Pour un avion sélectionné :
\begin{itemize}
    \item \textbf{W/S} : Augmenter/Diminuer la vitesse
    \item \textbf{A/D} : Tourner à gauche/droite
    \item \textbf{Q/E} : Monter/Descendre
    \item \textbf{B} : Freins
    \item \textbf{G} : Train d'atterrissage
\end{itemize}

\begin{tcolorbox}[tip]
Utilisez la touche Tab pour basculer entre les différents avions en vol.
\end{tcolorbox}

\section{Fonctionnalités avancées}

\subsection{Gestion du trafic aérien}
\begin{itemize}
    \item Sélection multiple d'avions
    \item Gestion des priorités
    \item Résolution automatique des conflits
\end{itemize}

\subsection{Conditions météorologiques}
Le simulateur prend en compte :
\begin{itemize}
    \item Direction et force du vent
    \item Visibilité
    \item Précipitations
\end{itemize}

\begin{tcolorbox}[warning]
Les conditions météorologiques affectent significativement le comportement des avions.
\end{tcolorbox}

\section{Dépannage}

\subsection{Problèmes courants}
\begin{itemize}
    \item \textbf{L'application ne démarre pas} :
    \begin{itemize}
        \item Vérifiez votre version de Java
        \item Essayez de lancer en ligne de commande pour voir les messages d'erreur
    \end{itemize}
    \item \textbf{Performances faibles} :
    \begin{itemize}
        \item Réduisez le nombre d'avions simultanés
        \item Diminuez la qualité graphique
        \item Fermez les applications en arrière-plan
    \end{itemize}
\end{itemize}

\end{document} 